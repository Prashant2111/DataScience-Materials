
% Default to the notebook output style

    


% Inherit from the specified cell style.




    
\documentclass[11pt]{article}

    
    
    \usepackage[T1]{fontenc}
    % Nicer default font (+ math font) than Computer Modern for most use cases
    \usepackage{mathpazo}

    % Basic figure setup, for now with no caption control since it's done
    % automatically by Pandoc (which extracts ![](path) syntax from Markdown).
    \usepackage{graphicx}
    % We will generate all images so they have a width \maxwidth. This means
    % that they will get their normal width if they fit onto the page, but
    % are scaled down if they would overflow the margins.
    \makeatletter
    \def\maxwidth{\ifdim\Gin@nat@width>\linewidth\linewidth
    \else\Gin@nat@width\fi}
    \makeatother
    \let\Oldincludegraphics\includegraphics
    % Set max figure width to be 80% of text width, for now hardcoded.
    \renewcommand{\includegraphics}[1]{\Oldincludegraphics[width=.8\maxwidth]{#1}}
    % Ensure that by default, figures have no caption (until we provide a
    % proper Figure object with a Caption API and a way to capture that
    % in the conversion process - todo).
    \usepackage{caption}
    \DeclareCaptionLabelFormat{nolabel}{}
    \captionsetup{labelformat=nolabel}

    \usepackage{adjustbox} % Used to constrain images to a maximum size 
    \usepackage{xcolor} % Allow colors to be defined
    \usepackage{enumerate} % Needed for markdown enumerations to work
    \usepackage{geometry} % Used to adjust the document margins
    \usepackage{amsmath} % Equations
    \usepackage{amssymb} % Equations
    \usepackage{textcomp} % defines textquotesingle
    % Hack from http://tex.stackexchange.com/a/47451/13684:
    \AtBeginDocument{%
        \def\PYZsq{\textquotesingle}% Upright quotes in Pygmentized code
    }
    \usepackage{upquote} % Upright quotes for verbatim code
    \usepackage{eurosym} % defines \euro
    \usepackage[mathletters]{ucs} % Extended unicode (utf-8) support
    \usepackage[utf8x]{inputenc} % Allow utf-8 characters in the tex document
    \usepackage{fancyvrb} % verbatim replacement that allows latex
    \usepackage{grffile} % extends the file name processing of package graphics 
                         % to support a larger range 
    % The hyperref package gives us a pdf with properly built
    % internal navigation ('pdf bookmarks' for the table of contents,
    % internal cross-reference links, web links for URLs, etc.)
    \usepackage{hyperref}
    \usepackage{longtable} % longtable support required by pandoc >1.10
    \usepackage{booktabs}  % table support for pandoc > 1.12.2
    \usepackage[inline]{enumitem} % IRkernel/repr support (it uses the enumerate* environment)
    \usepackage[normalem]{ulem} % ulem is needed to support strikethroughs (\sout)
                                % normalem makes italics be italics, not underlines
    

    
    
    % Colors for the hyperref package
    \definecolor{urlcolor}{rgb}{0,.145,.698}
    \definecolor{linkcolor}{rgb}{.71,0.21,0.01}
    \definecolor{citecolor}{rgb}{.12,.54,.11}

    % ANSI colors
    \definecolor{ansi-black}{HTML}{3E424D}
    \definecolor{ansi-black-intense}{HTML}{282C36}
    \definecolor{ansi-red}{HTML}{E75C58}
    \definecolor{ansi-red-intense}{HTML}{B22B31}
    \definecolor{ansi-green}{HTML}{00A250}
    \definecolor{ansi-green-intense}{HTML}{007427}
    \definecolor{ansi-yellow}{HTML}{DDB62B}
    \definecolor{ansi-yellow-intense}{HTML}{B27D12}
    \definecolor{ansi-blue}{HTML}{208FFB}
    \definecolor{ansi-blue-intense}{HTML}{0065CA}
    \definecolor{ansi-magenta}{HTML}{D160C4}
    \definecolor{ansi-magenta-intense}{HTML}{A03196}
    \definecolor{ansi-cyan}{HTML}{60C6C8}
    \definecolor{ansi-cyan-intense}{HTML}{258F8F}
    \definecolor{ansi-white}{HTML}{C5C1B4}
    \definecolor{ansi-white-intense}{HTML}{A1A6B2}

    % commands and environments needed by pandoc snippets
    % extracted from the output of `pandoc -s`
    \providecommand{\tightlist}{%
      \setlength{\itemsep}{0pt}\setlength{\parskip}{0pt}}
    \DefineVerbatimEnvironment{Highlighting}{Verbatim}{commandchars=\\\{\}}
    % Add ',fontsize=\small' for more characters per line
    \newenvironment{Shaded}{}{}
    \newcommand{\KeywordTok}[1]{\textcolor[rgb]{0.00,0.44,0.13}{\textbf{{#1}}}}
    \newcommand{\DataTypeTok}[1]{\textcolor[rgb]{0.56,0.13,0.00}{{#1}}}
    \newcommand{\DecValTok}[1]{\textcolor[rgb]{0.25,0.63,0.44}{{#1}}}
    \newcommand{\BaseNTok}[1]{\textcolor[rgb]{0.25,0.63,0.44}{{#1}}}
    \newcommand{\FloatTok}[1]{\textcolor[rgb]{0.25,0.63,0.44}{{#1}}}
    \newcommand{\CharTok}[1]{\textcolor[rgb]{0.25,0.44,0.63}{{#1}}}
    \newcommand{\StringTok}[1]{\textcolor[rgb]{0.25,0.44,0.63}{{#1}}}
    \newcommand{\CommentTok}[1]{\textcolor[rgb]{0.38,0.63,0.69}{\textit{{#1}}}}
    \newcommand{\OtherTok}[1]{\textcolor[rgb]{0.00,0.44,0.13}{{#1}}}
    \newcommand{\AlertTok}[1]{\textcolor[rgb]{1.00,0.00,0.00}{\textbf{{#1}}}}
    \newcommand{\FunctionTok}[1]{\textcolor[rgb]{0.02,0.16,0.49}{{#1}}}
    \newcommand{\RegionMarkerTok}[1]{{#1}}
    \newcommand{\ErrorTok}[1]{\textcolor[rgb]{1.00,0.00,0.00}{\textbf{{#1}}}}
    \newcommand{\NormalTok}[1]{{#1}}
    
    % Additional commands for more recent versions of Pandoc
    \newcommand{\ConstantTok}[1]{\textcolor[rgb]{0.53,0.00,0.00}{{#1}}}
    \newcommand{\SpecialCharTok}[1]{\textcolor[rgb]{0.25,0.44,0.63}{{#1}}}
    \newcommand{\VerbatimStringTok}[1]{\textcolor[rgb]{0.25,0.44,0.63}{{#1}}}
    \newcommand{\SpecialStringTok}[1]{\textcolor[rgb]{0.73,0.40,0.53}{{#1}}}
    \newcommand{\ImportTok}[1]{{#1}}
    \newcommand{\DocumentationTok}[1]{\textcolor[rgb]{0.73,0.13,0.13}{\textit{{#1}}}}
    \newcommand{\AnnotationTok}[1]{\textcolor[rgb]{0.38,0.63,0.69}{\textbf{\textit{{#1}}}}}
    \newcommand{\CommentVarTok}[1]{\textcolor[rgb]{0.38,0.63,0.69}{\textbf{\textit{{#1}}}}}
    \newcommand{\VariableTok}[1]{\textcolor[rgb]{0.10,0.09,0.49}{{#1}}}
    \newcommand{\ControlFlowTok}[1]{\textcolor[rgb]{0.00,0.44,0.13}{\textbf{{#1}}}}
    \newcommand{\OperatorTok}[1]{\textcolor[rgb]{0.40,0.40,0.40}{{#1}}}
    \newcommand{\BuiltInTok}[1]{{#1}}
    \newcommand{\ExtensionTok}[1]{{#1}}
    \newcommand{\PreprocessorTok}[1]{\textcolor[rgb]{0.74,0.48,0.00}{{#1}}}
    \newcommand{\AttributeTok}[1]{\textcolor[rgb]{0.49,0.56,0.16}{{#1}}}
    \newcommand{\InformationTok}[1]{\textcolor[rgb]{0.38,0.63,0.69}{\textbf{\textit{{#1}}}}}
    \newcommand{\WarningTok}[1]{\textcolor[rgb]{0.38,0.63,0.69}{\textbf{\textit{{#1}}}}}
    
    
    % Define a nice break command that doesn't care if a line doesn't already
    % exist.
    \def\br{\hspace*{\fill} \\* }
    % Math Jax compatability definitions
    \def\gt{>}
    \def\lt{<}
    % Document parameters
    \title{EDA\_day1}
    
    
    

    % Pygments definitions
    
\makeatletter
\def\PY@reset{\let\PY@it=\relax \let\PY@bf=\relax%
    \let\PY@ul=\relax \let\PY@tc=\relax%
    \let\PY@bc=\relax \let\PY@ff=\relax}
\def\PY@tok#1{\csname PY@tok@#1\endcsname}
\def\PY@toks#1+{\ifx\relax#1\empty\else%
    \PY@tok{#1}\expandafter\PY@toks\fi}
\def\PY@do#1{\PY@bc{\PY@tc{\PY@ul{%
    \PY@it{\PY@bf{\PY@ff{#1}}}}}}}
\def\PY#1#2{\PY@reset\PY@toks#1+\relax+\PY@do{#2}}

\expandafter\def\csname PY@tok@gd\endcsname{\def\PY@tc##1{\textcolor[rgb]{0.63,0.00,0.00}{##1}}}
\expandafter\def\csname PY@tok@gu\endcsname{\let\PY@bf=\textbf\def\PY@tc##1{\textcolor[rgb]{0.50,0.00,0.50}{##1}}}
\expandafter\def\csname PY@tok@gt\endcsname{\def\PY@tc##1{\textcolor[rgb]{0.00,0.27,0.87}{##1}}}
\expandafter\def\csname PY@tok@gs\endcsname{\let\PY@bf=\textbf}
\expandafter\def\csname PY@tok@gr\endcsname{\def\PY@tc##1{\textcolor[rgb]{1.00,0.00,0.00}{##1}}}
\expandafter\def\csname PY@tok@cm\endcsname{\let\PY@it=\textit\def\PY@tc##1{\textcolor[rgb]{0.25,0.50,0.50}{##1}}}
\expandafter\def\csname PY@tok@vg\endcsname{\def\PY@tc##1{\textcolor[rgb]{0.10,0.09,0.49}{##1}}}
\expandafter\def\csname PY@tok@vi\endcsname{\def\PY@tc##1{\textcolor[rgb]{0.10,0.09,0.49}{##1}}}
\expandafter\def\csname PY@tok@vm\endcsname{\def\PY@tc##1{\textcolor[rgb]{0.10,0.09,0.49}{##1}}}
\expandafter\def\csname PY@tok@mh\endcsname{\def\PY@tc##1{\textcolor[rgb]{0.40,0.40,0.40}{##1}}}
\expandafter\def\csname PY@tok@cs\endcsname{\let\PY@it=\textit\def\PY@tc##1{\textcolor[rgb]{0.25,0.50,0.50}{##1}}}
\expandafter\def\csname PY@tok@ge\endcsname{\let\PY@it=\textit}
\expandafter\def\csname PY@tok@vc\endcsname{\def\PY@tc##1{\textcolor[rgb]{0.10,0.09,0.49}{##1}}}
\expandafter\def\csname PY@tok@il\endcsname{\def\PY@tc##1{\textcolor[rgb]{0.40,0.40,0.40}{##1}}}
\expandafter\def\csname PY@tok@go\endcsname{\def\PY@tc##1{\textcolor[rgb]{0.53,0.53,0.53}{##1}}}
\expandafter\def\csname PY@tok@cp\endcsname{\def\PY@tc##1{\textcolor[rgb]{0.74,0.48,0.00}{##1}}}
\expandafter\def\csname PY@tok@gi\endcsname{\def\PY@tc##1{\textcolor[rgb]{0.00,0.63,0.00}{##1}}}
\expandafter\def\csname PY@tok@gh\endcsname{\let\PY@bf=\textbf\def\PY@tc##1{\textcolor[rgb]{0.00,0.00,0.50}{##1}}}
\expandafter\def\csname PY@tok@ni\endcsname{\let\PY@bf=\textbf\def\PY@tc##1{\textcolor[rgb]{0.60,0.60,0.60}{##1}}}
\expandafter\def\csname PY@tok@nl\endcsname{\def\PY@tc##1{\textcolor[rgb]{0.63,0.63,0.00}{##1}}}
\expandafter\def\csname PY@tok@nn\endcsname{\let\PY@bf=\textbf\def\PY@tc##1{\textcolor[rgb]{0.00,0.00,1.00}{##1}}}
\expandafter\def\csname PY@tok@no\endcsname{\def\PY@tc##1{\textcolor[rgb]{0.53,0.00,0.00}{##1}}}
\expandafter\def\csname PY@tok@na\endcsname{\def\PY@tc##1{\textcolor[rgb]{0.49,0.56,0.16}{##1}}}
\expandafter\def\csname PY@tok@nb\endcsname{\def\PY@tc##1{\textcolor[rgb]{0.00,0.50,0.00}{##1}}}
\expandafter\def\csname PY@tok@nc\endcsname{\let\PY@bf=\textbf\def\PY@tc##1{\textcolor[rgb]{0.00,0.00,1.00}{##1}}}
\expandafter\def\csname PY@tok@nd\endcsname{\def\PY@tc##1{\textcolor[rgb]{0.67,0.13,1.00}{##1}}}
\expandafter\def\csname PY@tok@ne\endcsname{\let\PY@bf=\textbf\def\PY@tc##1{\textcolor[rgb]{0.82,0.25,0.23}{##1}}}
\expandafter\def\csname PY@tok@nf\endcsname{\def\PY@tc##1{\textcolor[rgb]{0.00,0.00,1.00}{##1}}}
\expandafter\def\csname PY@tok@si\endcsname{\let\PY@bf=\textbf\def\PY@tc##1{\textcolor[rgb]{0.73,0.40,0.53}{##1}}}
\expandafter\def\csname PY@tok@s2\endcsname{\def\PY@tc##1{\textcolor[rgb]{0.73,0.13,0.13}{##1}}}
\expandafter\def\csname PY@tok@nt\endcsname{\let\PY@bf=\textbf\def\PY@tc##1{\textcolor[rgb]{0.00,0.50,0.00}{##1}}}
\expandafter\def\csname PY@tok@nv\endcsname{\def\PY@tc##1{\textcolor[rgb]{0.10,0.09,0.49}{##1}}}
\expandafter\def\csname PY@tok@s1\endcsname{\def\PY@tc##1{\textcolor[rgb]{0.73,0.13,0.13}{##1}}}
\expandafter\def\csname PY@tok@dl\endcsname{\def\PY@tc##1{\textcolor[rgb]{0.73,0.13,0.13}{##1}}}
\expandafter\def\csname PY@tok@ch\endcsname{\let\PY@it=\textit\def\PY@tc##1{\textcolor[rgb]{0.25,0.50,0.50}{##1}}}
\expandafter\def\csname PY@tok@m\endcsname{\def\PY@tc##1{\textcolor[rgb]{0.40,0.40,0.40}{##1}}}
\expandafter\def\csname PY@tok@gp\endcsname{\let\PY@bf=\textbf\def\PY@tc##1{\textcolor[rgb]{0.00,0.00,0.50}{##1}}}
\expandafter\def\csname PY@tok@sh\endcsname{\def\PY@tc##1{\textcolor[rgb]{0.73,0.13,0.13}{##1}}}
\expandafter\def\csname PY@tok@ow\endcsname{\let\PY@bf=\textbf\def\PY@tc##1{\textcolor[rgb]{0.67,0.13,1.00}{##1}}}
\expandafter\def\csname PY@tok@sx\endcsname{\def\PY@tc##1{\textcolor[rgb]{0.00,0.50,0.00}{##1}}}
\expandafter\def\csname PY@tok@bp\endcsname{\def\PY@tc##1{\textcolor[rgb]{0.00,0.50,0.00}{##1}}}
\expandafter\def\csname PY@tok@c1\endcsname{\let\PY@it=\textit\def\PY@tc##1{\textcolor[rgb]{0.25,0.50,0.50}{##1}}}
\expandafter\def\csname PY@tok@fm\endcsname{\def\PY@tc##1{\textcolor[rgb]{0.00,0.00,1.00}{##1}}}
\expandafter\def\csname PY@tok@o\endcsname{\def\PY@tc##1{\textcolor[rgb]{0.40,0.40,0.40}{##1}}}
\expandafter\def\csname PY@tok@kc\endcsname{\let\PY@bf=\textbf\def\PY@tc##1{\textcolor[rgb]{0.00,0.50,0.00}{##1}}}
\expandafter\def\csname PY@tok@c\endcsname{\let\PY@it=\textit\def\PY@tc##1{\textcolor[rgb]{0.25,0.50,0.50}{##1}}}
\expandafter\def\csname PY@tok@mf\endcsname{\def\PY@tc##1{\textcolor[rgb]{0.40,0.40,0.40}{##1}}}
\expandafter\def\csname PY@tok@err\endcsname{\def\PY@bc##1{\setlength{\fboxsep}{0pt}\fcolorbox[rgb]{1.00,0.00,0.00}{1,1,1}{\strut ##1}}}
\expandafter\def\csname PY@tok@mb\endcsname{\def\PY@tc##1{\textcolor[rgb]{0.40,0.40,0.40}{##1}}}
\expandafter\def\csname PY@tok@ss\endcsname{\def\PY@tc##1{\textcolor[rgb]{0.10,0.09,0.49}{##1}}}
\expandafter\def\csname PY@tok@sr\endcsname{\def\PY@tc##1{\textcolor[rgb]{0.73,0.40,0.53}{##1}}}
\expandafter\def\csname PY@tok@mo\endcsname{\def\PY@tc##1{\textcolor[rgb]{0.40,0.40,0.40}{##1}}}
\expandafter\def\csname PY@tok@kd\endcsname{\let\PY@bf=\textbf\def\PY@tc##1{\textcolor[rgb]{0.00,0.50,0.00}{##1}}}
\expandafter\def\csname PY@tok@mi\endcsname{\def\PY@tc##1{\textcolor[rgb]{0.40,0.40,0.40}{##1}}}
\expandafter\def\csname PY@tok@kn\endcsname{\let\PY@bf=\textbf\def\PY@tc##1{\textcolor[rgb]{0.00,0.50,0.00}{##1}}}
\expandafter\def\csname PY@tok@cpf\endcsname{\let\PY@it=\textit\def\PY@tc##1{\textcolor[rgb]{0.25,0.50,0.50}{##1}}}
\expandafter\def\csname PY@tok@kr\endcsname{\let\PY@bf=\textbf\def\PY@tc##1{\textcolor[rgb]{0.00,0.50,0.00}{##1}}}
\expandafter\def\csname PY@tok@s\endcsname{\def\PY@tc##1{\textcolor[rgb]{0.73,0.13,0.13}{##1}}}
\expandafter\def\csname PY@tok@kp\endcsname{\def\PY@tc##1{\textcolor[rgb]{0.00,0.50,0.00}{##1}}}
\expandafter\def\csname PY@tok@w\endcsname{\def\PY@tc##1{\textcolor[rgb]{0.73,0.73,0.73}{##1}}}
\expandafter\def\csname PY@tok@kt\endcsname{\def\PY@tc##1{\textcolor[rgb]{0.69,0.00,0.25}{##1}}}
\expandafter\def\csname PY@tok@sc\endcsname{\def\PY@tc##1{\textcolor[rgb]{0.73,0.13,0.13}{##1}}}
\expandafter\def\csname PY@tok@sb\endcsname{\def\PY@tc##1{\textcolor[rgb]{0.73,0.13,0.13}{##1}}}
\expandafter\def\csname PY@tok@sa\endcsname{\def\PY@tc##1{\textcolor[rgb]{0.73,0.13,0.13}{##1}}}
\expandafter\def\csname PY@tok@k\endcsname{\let\PY@bf=\textbf\def\PY@tc##1{\textcolor[rgb]{0.00,0.50,0.00}{##1}}}
\expandafter\def\csname PY@tok@se\endcsname{\let\PY@bf=\textbf\def\PY@tc##1{\textcolor[rgb]{0.73,0.40,0.13}{##1}}}
\expandafter\def\csname PY@tok@sd\endcsname{\let\PY@it=\textit\def\PY@tc##1{\textcolor[rgb]{0.73,0.13,0.13}{##1}}}

\def\PYZbs{\char`\\}
\def\PYZus{\char`\_}
\def\PYZob{\char`\{}
\def\PYZcb{\char`\}}
\def\PYZca{\char`\^}
\def\PYZam{\char`\&}
\def\PYZlt{\char`\<}
\def\PYZgt{\char`\>}
\def\PYZsh{\char`\#}
\def\PYZpc{\char`\%}
\def\PYZdl{\char`\$}
\def\PYZhy{\char`\-}
\def\PYZsq{\char`\'}
\def\PYZdq{\char`\"}
\def\PYZti{\char`\~}
% for compatibility with earlier versions
\def\PYZat{@}
\def\PYZlb{[}
\def\PYZrb{]}
\makeatother


    % Exact colors from NB
    \definecolor{incolor}{rgb}{0.0, 0.0, 0.5}
    \definecolor{outcolor}{rgb}{0.545, 0.0, 0.0}



    
    % Prevent overflowing lines due to hard-to-break entities
    \sloppy 
    % Setup hyperref package
    \hypersetup{
      breaklinks=true,  % so long urls are correctly broken across lines
      colorlinks=true,
      urlcolor=urlcolor,
      linkcolor=linkcolor,
      citecolor=citecolor,
      }
    % Slightly bigger margins than the latex defaults
    
    \geometry{verbose,tmargin=1in,bmargin=1in,lmargin=1in,rmargin=1in}
    
    

    \begin{document}
    
    
    \maketitle
    
    

    
    \section{Introduction to Pandas}\label{introduction-to-pandas}

    Importing the Pandas package

    \begin{Verbatim}[commandchars=\\\{\}]
{\color{incolor}In [{\color{incolor} }]:} \PY{k+kn}{import} \PY{n+nn}{pandas} \PY{k+kn}{as} \PY{n+nn}{pd} \PY{c+c1}{\PYZsh{}this will import pandas into your workspace}
        \PY{k+kn}{import} \PY{n+nn}{numpy} \PY{k+kn}{as} \PY{n+nn}{np} 
\end{Verbatim}


    \textbf{Data Structures in pandas}

There are two basic data structures in pandas: Series and DataFrame

    \section{Series}\label{series}

    It is similar to a NumPy 1-dimensional array. In addition to the values
that are specified by the programmer, pandas attaches a label to each of
the values. If the labels are not provided by the programmer, then
pandas assigns labels ( 0 for first element, 1 for second element and so
on). A benefit of assigning labels to data values is that it becomes
easier to perform manipulations on the dataset as the whole dataset
becomes more of a dictionary where each value is associated with a
label.

    \section{Lets create our first python
series}\label{lets-create-our-first-python-series}

    \begin{Verbatim}[commandchars=\\\{\}]
{\color{incolor}In [{\color{incolor} }]:} \PY{n}{series1} \PY{o}{=} \PY{n}{pd}\PY{o}{.}\PY{n}{Series}\PY{p}{(}\PY{p}{[}\PY{l+m+mi}{10}\PY{p}{,}\PY{l+m+mi}{20}\PY{p}{,}\PY{l+m+mi}{30}\PY{p}{,}\PY{l+m+mi}{40}\PY{p}{]}\PY{p}{)} \PY{c+c1}{\PYZsh{}we have used a list to create a series.}
        \PY{k}{print}\PY{p}{(}\PY{n}{series1}\PY{p}{)}
\end{Verbatim}


    \section{Lets know find out the values that are there in the
series}\label{lets-know-find-out-the-values-that-are-there-in-the-series}

    \begin{Verbatim}[commandchars=\\\{\}]
{\color{incolor}In [{\color{incolor} }]:} \PY{n}{series1}\PY{o}{.}\PY{n}{values}
\end{Verbatim}


    \section{Lets find out the index number that are there in the
series}\label{lets-find-out-the-index-number-that-are-there-in-the-series}

    \begin{Verbatim}[commandchars=\\\{\}]
{\color{incolor}In [{\color{incolor} }]:} \PY{n}{series1}\PY{o}{.}\PY{n}{index} \PY{c+c1}{\PYZsh{}This would print the starting index and ending index}
\end{Verbatim}


    \section{If you want to specify custom index values rather than the
default ones provided, you can do so using the following
command}\label{if-you-want-to-specify-custom-index-values-rather-than-the-default-ones-provided-you-can-do-so-using-the-following-command}

    \begin{Verbatim}[commandchars=\\\{\}]
{\color{incolor}In [{\color{incolor} }]:} \PY{n}{series2} \PY{o}{=} \PY{n}{pd}\PY{o}{.}\PY{n}{Series}\PY{p}{(}\PY{p}{[}\PY{l+m+mi}{10}\PY{p}{,}\PY{l+m+mi}{20}\PY{p}{,}\PY{l+m+mi}{30}\PY{p}{,}\PY{l+m+mi}{40}\PY{p}{,}\PY{l+m+mi}{50}\PY{p}{]}\PY{p}{,} \PY{n}{index}\PY{o}{=}\PY{p}{[}\PY{l+s+s1}{\PYZsq{}}\PY{l+s+s1}{one}\PY{l+s+s1}{\PYZsq{}}\PY{p}{,}\PY{l+s+s1}{\PYZsq{}}\PY{l+s+s1}{two}\PY{l+s+s1}{\PYZsq{}}\PY{p}{,}\PY{l+s+s1}{\PYZsq{}}\PY{l+s+s1}{three}\PY{l+s+s1}{\PYZsq{}}\PY{p}{,}\PY{l+s+s1}{\PYZsq{}}\PY{l+s+s1}{four}\PY{l+s+s1}{\PYZsq{}}\PY{p}{,}\PY{l+s+s1}{\PYZsq{}}\PY{l+s+s1}{five}\PY{l+s+s1}{\PYZsq{}}\PY{p}{]}\PY{p}{)}
        \PY{n}{series2}
\end{Verbatim}


    \section{Lets print the element which is there in the position
2}\label{lets-print-the-element-which-is-there-in-the-position-2}

    \begin{Verbatim}[commandchars=\\\{\}]
{\color{incolor}In [{\color{incolor} }]:} \PY{n}{series2}\PY{p}{[}\PY{l+m+mi}{2}\PY{p}{]}
\end{Verbatim}


    \section{Lets retrive the element using index
number}\label{lets-retrive-the-element-using-index-number}

    \begin{Verbatim}[commandchars=\\\{\}]
{\color{incolor}In [{\color{incolor} }]:} \PY{n}{series2} \PY{o}{=} \PY{n}{pd}\PY{o}{.}\PY{n}{Series}\PY{p}{(}\PY{p}{[}\PY{l+m+mi}{10}\PY{p}{,}\PY{l+m+mi}{20}\PY{p}{,}\PY{l+m+mi}{30}\PY{p}{,}\PY{l+m+mi}{40}\PY{p}{,}\PY{l+m+mi}{50}\PY{p}{]}\PY{p}{,} \PY{n}{index}\PY{o}{=}\PY{p}{[}\PY{l+s+s1}{\PYZsq{}}\PY{l+s+s1}{one}\PY{l+s+s1}{\PYZsq{}}\PY{p}{,}\PY{l+s+s1}{\PYZsq{}}\PY{l+s+s1}{two}\PY{l+s+s1}{\PYZsq{}}\PY{p}{,}\PY{l+s+s1}{\PYZsq{}}\PY{l+s+s1}{three}\PY{l+s+s1}{\PYZsq{}}\PY{p}{,}\PY{l+s+s1}{\PYZsq{}}\PY{l+s+s1}{four}\PY{l+s+s1}{\PYZsq{}}\PY{p}{,}\PY{l+s+s1}{\PYZsq{}}\PY{l+s+s1}{five}\PY{l+s+s1}{\PYZsq{}}\PY{p}{]}\PY{p}{)}
        \PY{k}{print}\PY{p}{(}\PY{n}{series2}\PY{p}{)}
        \PY{n}{series2}\PY{p}{[}\PY{l+s+s1}{\PYZsq{}}\PY{l+s+s1}{three}\PY{l+s+s1}{\PYZsq{}}\PY{p}{]}
\end{Verbatim}


    \section{Lets access multiple
elements}\label{lets-access-multiple-elements}

    \begin{Verbatim}[commandchars=\\\{\}]
{\color{incolor}In [{\color{incolor} }]:} \PY{n}{series2}\PY{p}{[}\PY{p}{[}\PY{l+s+s1}{\PYZsq{}}\PY{l+s+s1}{one}\PY{l+s+s1}{\PYZsq{}}\PY{p}{,} \PY{l+s+s1}{\PYZsq{}}\PY{l+s+s1}{three}\PY{l+s+s1}{\PYZsq{}}\PY{p}{,} \PY{l+s+s1}{\PYZsq{}}\PY{l+s+s1}{five}\PY{l+s+s1}{\PYZsq{}}\PY{p}{]}\PY{p}{]}
\end{Verbatim}


    \section{Lets add "4" to each element of the series (math
operations)}\label{lets-add-4-to-each-element-of-the-series-math-operations}

    \begin{Verbatim}[commandchars=\\\{\}]
{\color{incolor}In [{\color{incolor} }]:} \PY{n}{series2} \PY{o}{+} \PY{l+m+mi}{4}
\end{Verbatim}


    \section{Lets subset the entire series whose value is greater than
30}\label{lets-subset-the-entire-series-whose-value-is-greater-than-30}

    \begin{Verbatim}[commandchars=\\\{\}]
{\color{incolor}In [{\color{incolor} }]:} \PY{n}{series2}\PY{p}{[}\PY{n}{series2}\PY{o}{\PYZgt{}}\PY{l+m+mi}{30}\PY{p}{]}
\end{Verbatim}


    \section{The usual way of creating and initializing dictonary in
Python}\label{the-usual-way-of-creating-and-initializing-dictonary-in-python}

    \begin{Verbatim}[commandchars=\\\{\}]
{\color{incolor}In [{\color{incolor} }]:} \PY{n}{mydict} \PY{o}{=}\PY{p}{\PYZob{}}
          \PY{l+s+s2}{\PYZdq{}}\PY{l+s+s2}{company}\PY{l+s+s2}{\PYZdq{}}\PY{p}{:} \PY{l+s+s2}{\PYZdq{}}\PY{l+s+s2}{Toyota}\PY{l+s+s2}{\PYZdq{}}\PY{p}{,}
          \PY{l+s+s2}{\PYZdq{}}\PY{l+s+s2}{model}\PY{l+s+s2}{\PYZdq{}}\PY{p}{:} \PY{l+s+s2}{\PYZdq{}}\PY{l+s+s2}{Corolla}\PY{l+s+s2}{\PYZdq{}}\PY{p}{,}
          \PY{l+s+s2}{\PYZdq{}}\PY{l+s+s2}{year}\PY{l+s+s2}{\PYZdq{}}\PY{p}{:} \PY{l+m+mi}{1994}
        \PY{p}{\PYZcb{}}
        \PY{k}{print}\PY{p}{(}\PY{n}{mydict}\PY{p}{)}
\end{Verbatim}


    \section{A quick way to initialize a
dictionary}\label{a-quick-way-to-initialize-a-dictionary}

    \begin{Verbatim}[commandchars=\\\{\}]
{\color{incolor}In [{\color{incolor} }]:} \PY{n}{myl} \PY{o}{=} \PY{p}{[}\PY{l+m+mi}{1}\PY{p}{,}\PY{l+m+mi}{2}\PY{p}{,}\PY{l+m+mi}{3}\PY{p}{,}\PY{l+m+mi}{4}\PY{p}{,}\PY{l+m+mi}{5}\PY{p}{]}
        \PY{n}{myd} \PY{o}{=} \PY{n+nb}{dict}\PY{p}{(}\PY{p}{(}\PY{n}{x}\PY{p}{,}\PY{l+m+mi}{0}\PY{p}{)} \PY{k}{for} \PY{n}{x} \PY{o+ow}{in} \PY{n}{myl}\PY{p}{)} \PY{c+c1}{\PYZsh{}x represents the key element in the dictionary }
        \PY{k}{print}\PY{p}{(}\PY{n}{myd}\PY{p}{)}
        \PY{n+nb}{type}\PY{p}{(}\PY{n}{myd}\PY{p}{)}
\end{Verbatim}


    \section{WAP - In class exe :Write code to print a dictionary where the
keys are numbers between 1 and 15 (both included) and the values are
square of
keys.}\label{wap---in-class-exe-write-code-to-print-a-dictionary-where-the-keys-are-numbers-between-1-and-15-both-included-and-the-values-are-square-of-keys.}

\section{Sample Dictionary \{1: 1, 2: 4, 3: 9, 4: 16, 5: 25, 6: 36, 7:
49, 8: 64, 9: 81, 10: 100, 11: 121, 12: 144, 13: 169, 14: 196, 15:
225\}}\label{sample-dictionary-1-1-2-4-3-9-4-16-5-25-6-36-7-49-8-64-9-81-10-100-11-121-12-144-13-169-14-196-15-225}

    \subsubsection{Creating a tabular view of data using dictonary and
series}\label{creating-a-tabular-view-of-data-using-dictonary-and-series}

    \begin{Verbatim}[commandchars=\\\{\}]
{\color{incolor}In [{\color{incolor} }]:} \PY{n}{myd} \PY{o}{=} \PY{p}{\PYZob{}}\PY{l+s+s2}{\PYZdq{}}\PY{l+s+s2}{Orange}\PY{l+s+s2}{\PYZdq{}}\PY{p}{:}\PY{l+m+mi}{1}\PY{p}{,}\PY{l+s+s2}{\PYZdq{}}\PY{l+s+s2}{Apple}\PY{l+s+s2}{\PYZdq{}}\PY{p}{:}\PY{l+m+mi}{2}\PY{p}{,}\PY{l+s+s2}{\PYZdq{}}\PY{l+s+s2}{Grapes}\PY{l+s+s2}{\PYZdq{}}\PY{p}{:}\PY{l+m+mi}{3}\PY{p}{,} \PY{l+s+s2}{\PYZdq{}}\PY{l+s+s2}{Apple}\PY{l+s+s2}{\PYZdq{}}\PY{p}{:}\PY{l+m+mi}{1}\PY{p}{\PYZcb{}} \PY{c+c1}{\PYZsh{} myd is a dictonary type }
        \PY{n}{mytab} \PY{o}{=} \PY{n}{pd}\PY{o}{.}\PY{n}{Series}\PY{p}{(}\PY{n}{myd}\PY{p}{)}
        \PY{k}{print}\PY{p}{(}\PY{n}{mytab}\PY{p}{)} \PY{c+c1}{\PYZsh{} note that the series data will be sorted }
        \PY{n+nb}{type}\PY{p}{(}\PY{n}{mytab}\PY{p}{)}
        \PY{n}{mytab}\PY{p}{[}\PY{l+m+mi}{1}\PY{p}{]} \PY{c+c1}{\PYZsh{}This will print only the values from the series ! }
        \PY{n}{mytab}\PY{o}{.}\PY{n}{index}
\end{Verbatim}


    \section{Lets create a dataset for a firm which contains its profits on
the yearly
basis}\label{lets-create-a-dataset-for-a-firm-which-contains-its-profits-on-the-yearly-basis}

    \begin{Verbatim}[commandchars=\\\{\}]
{\color{incolor}In [{\color{incolor} }]:} \PY{n}{years} \PY{o}{=} \PY{p}{[}\PY{l+m+mi}{90}\PY{p}{,} \PY{l+m+mi}{91}\PY{p}{,} \PY{l+m+mi}{92}\PY{p}{,} \PY{l+m+mi}{93}\PY{p}{,} \PY{l+m+mi}{94}\PY{p}{,} \PY{l+m+mi}{95}\PY{p}{]} \PY{c+c1}{\PYZsh{} Note that the elements in this list match the }
        \PY{c+c1}{\PYZsh{}the keys in the below dictonary}
        \PY{n}{f1} \PY{o}{=} \PY{p}{\PYZob{}}\PY{l+m+mi}{90}\PY{p}{:}\PY{l+m+mi}{8}\PY{p}{,} \PY{l+m+mi}{91}\PY{p}{:}\PY{l+m+mi}{9}\PY{p}{,} \PY{l+m+mi}{92}\PY{p}{:}\PY{l+m+mi}{7}\PY{p}{,} \PY{l+m+mi}{93}\PY{p}{:}\PY{l+m+mi}{8}\PY{p}{,} \PY{l+m+mi}{94}\PY{p}{:}\PY{l+m+mi}{9}\PY{p}{,} \PY{l+m+mi}{95}\PY{p}{:}\PY{l+m+mi}{11}\PY{p}{\PYZcb{}}
        \PY{n}{firm1} \PY{o}{=} \PY{n}{pd}\PY{o}{.}\PY{n}{Series}\PY{p}{(}\PY{n}{f1}\PY{p}{,}\PY{n}{index}\PY{o}{=}\PY{n}{years}\PY{p}{)}
        \PY{k}{print}\PY{p}{(}\PY{n}{firm1}\PY{p}{)}
        \PY{n+nb}{type}\PY{p}{(}\PY{n}{firm1}\PY{p}{)}
\end{Verbatim}


    \section{Lets create a another dataset for a new firm called firm2 for
the above mentioned years based on the details available
below}\label{lets-create-a-another-dataset-for-a-new-firm-called-firm2-for-the-above-mentioned-years-based-on-the-details-available-below}

f2 = \{90:14,92:9, 93:13, 94:5\}

    \begin{Verbatim}[commandchars=\\\{\}]
{\color{incolor}In [{\color{incolor} }]:} \PY{n}{f2} \PY{o}{=} \PY{p}{\PYZob{}}\PY{l+m+mi}{90}\PY{p}{:}\PY{l+m+mi}{14}\PY{p}{,}\PY{l+m+mi}{92}\PY{p}{:}\PY{l+m+mi}{9}\PY{p}{,} \PY{l+m+mi}{93}\PY{p}{:}\PY{l+m+mi}{13}\PY{p}{,} \PY{l+m+mi}{94}\PY{p}{:}\PY{l+m+mi}{5}\PY{p}{\PYZcb{}}
        \PY{n}{firm2} \PY{o}{=} \PY{n}{pd}\PY{o}{.}\PY{n}{Series}\PY{p}{(}\PY{n}{f2}\PY{p}{,}\PY{n}{index}\PY{o}{=}\PY{n}{years}\PY{p}{)}
        \PY{n}{firm2}
\end{Verbatim}


    \section{Q) How did u get NAN in the above output
?}\label{q-how-did-u-get-nan-in-the-above-output}
# Note: We always do not get full data from the source or firm. In that case, there will be a missing values where the missing value would be considered as "NaN" which means a "Null value".
    \section{How to find missing values in
Pandas?}\label{how-to-find-missing-values-in-pandas}

    \begin{Verbatim}[commandchars=\\\{\}]
{\color{incolor}In [{\color{incolor} }]:} \PY{n}{pd}\PY{o}{.}\PY{n}{isnull}\PY{p}{(}\PY{n}{firm2}\PY{p}{)}
\end{Verbatim}


    \section{WAP - In class exe :Write code to concatenate following
dictionaries to create a new
one.}\label{wap---in-class-exe-write-code-to-concatenate-following-dictionaries-to-create-a-new-one.}

\section{Sample Dictionary : dic1=\{1:10, 2:20\} dic2=\{3:30, 4:40\}
dic3=\{5:50,6:60\}}\label{sample-dictionary-dic1110-220-dic2330-440-dic3550660}

\section{Expected Result : \{1: 10, 2: 20, 3: 30, 4: 40, 5: 50, 6:
60\}}\label{expected-result-1-10-2-20-3-30-4-40-5-50-6-60}

    \section{Data Frame}\label{data-frame}
DataFrame is a tabular data structure in which data is laid out in rows and column format (similar to a CSV or tables in SQL file), but it can also be used for higher dimensional data sets. The DataFrame object can contain homogenous and heterogenous values, and can be thought of as a logical extension of Series data structures. In contrast to Series, where there is one index, a DataFrame object has one index for column and one index for rows. This allows flexibility in accessing and manipulating data.
    \section{Lets create a Data Frame with multiple columns called Price,
Ticker and
Company.}\label{lets-create-a-data-frame-with-multiple-columns-called-price-ticker-and-company.}

    \begin{Verbatim}[commandchars=\\\{\}]
{\color{incolor}In [{\color{incolor} }]:} \PY{n}{data} \PY{o}{=} \PY{n}{pd}\PY{o}{.}\PY{n}{DataFrame}\PY{p}{(}\PY{p}{\PYZob{}}\PY{l+s+s1}{\PYZsq{}}\PY{l+s+s1}{price}\PY{l+s+s1}{\PYZsq{}}\PY{p}{:}\PY{p}{[}\PY{l+m+mi}{95}\PY{p}{,} \PY{l+m+mi}{25}\PY{p}{,} \PY{l+m+mi}{85}\PY{p}{,} \PY{l+m+mi}{41}\PY{p}{,} \PY{l+m+mi}{78}\PY{p}{]}\PY{p}{,}
                             \PY{l+s+s1}{\PYZsq{}}\PY{l+s+s1}{ticker}\PY{l+s+s1}{\PYZsq{}}\PY{p}{:}\PY{p}{[}\PY{l+s+s1}{\PYZsq{}}\PY{l+s+s1}{AXP}\PY{l+s+s1}{\PYZsq{}}\PY{p}{,} \PY{l+s+s1}{\PYZsq{}}\PY{l+s+s1}{CSCO}\PY{l+s+s1}{\PYZsq{}}\PY{p}{,} \PY{l+s+s1}{\PYZsq{}}\PY{l+s+s1}{DIS}\PY{l+s+s1}{\PYZsq{}}\PY{p}{,} \PY{l+s+s1}{\PYZsq{}}\PY{l+s+s1}{MSFT}\PY{l+s+s1}{\PYZsq{}}\PY{p}{,} \PY{l+s+s1}{\PYZsq{}}\PY{l+s+s1}{WMT}\PY{l+s+s1}{\PYZsq{}}\PY{p}{]}\PY{p}{,}
                             \PY{l+s+s1}{\PYZsq{}}\PY{l+s+s1}{company}\PY{l+s+s1}{\PYZsq{}}\PY{p}{:}\PY{p}{[}\PY{l+s+s1}{\PYZsq{}}\PY{l+s+s1}{American Express}\PY{l+s+s1}{\PYZsq{}}\PY{p}{,} \PY{l+s+s1}{\PYZsq{}}\PY{l+s+s1}{Cisco}\PY{l+s+s1}{\PYZsq{}}\PY{p}{,} \PY{l+s+s1}{\PYZsq{}}\PY{l+s+s1}{Walt Disney}\PY{l+s+s1}{\PYZsq{}}\PY{p}{,}\PY{l+s+s1}{\PYZsq{}}\PY{l+s+s1}{Microsoft}\PY{l+s+s1}{\PYZsq{}}\PY{p}{,} \PY{l+s+s1}{\PYZsq{}}\PY{l+s+s1}{Walmart}\PY{l+s+s1}{\PYZsq{}}\PY{p}{]}\PY{p}{\PYZcb{}}\PY{p}{)}
        \PY{n}{data}
\end{Verbatim}

#Note: If a column is passed with no values, it will simply have NaN values
    \section{How to access a specefic column from the data
frame?}\label{how-to-access-a-specefic-column-from-the-data-frame}

    \begin{Verbatim}[commandchars=\\\{\}]
{\color{incolor}In [{\color{incolor} }]:} \PY{n}{data}\PY{p}{[}\PY{l+s+s1}{\PYZsq{}}\PY{l+s+s1}{company}\PY{l+s+s1}{\PYZsq{}}\PY{p}{]}
\end{Verbatim}


    \section{How to access a specefic row from the data
frame?}\label{how-to-access-a-specefic-row-from-the-data-frame}

    \begin{Verbatim}[commandchars=\\\{\}]
{\color{incolor}In [{\color{incolor} }]:} \PY{n}{data}\PY{o}{.}\PY{n}{ix}\PY{p}{[}\PY{l+m+mi}{2}\PY{p}{]} \PY{c+c1}{\PYZsh{}Will print all the elements of second row }
\end{Verbatim}

#Note: ix is the shortform form for index.
    \section{How to add a new column in the data
frame?}\label{how-to-add-a-new-column-in-the-data-frame}

    \begin{Verbatim}[commandchars=\\\{\}]
{\color{incolor}In [{\color{incolor} }]:} \PY{n}{data}\PY{p}{[}\PY{l+s+s1}{\PYZsq{}}\PY{l+s+s1}{Year}\PY{l+s+s1}{\PYZsq{}}\PY{p}{]} \PY{o}{=} \PY{l+m+mi}{2014}
        \PY{n}{data}
\end{Verbatim}


    \section{WAP - In class exe : Create a new column "prices\_discount" in
the above data frame where the value should be
10\%}\label{wap---in-class-exe-create-a-new-column-prices_discount-in-the-above-data-frame-where-the-value-should-be-10}

\section{discounted from the original column "price". This change must
be implmented in
the}\label{discounted-from-the-original-column-price.-this-change-must-be-implmented-in-the}

\section{same data frame}\label{same-data-frame}

    \section{How to create a column and populate it with missing values(NaN)
?}\label{how-to-create-a-column-and-populate-it-with-missing-valuesnan}

    \begin{Verbatim}[commandchars=\\\{\}]
{\color{incolor}In [{\color{incolor} }]:} \PY{n}{data}\PY{p}{[}\PY{l+s+s1}{\PYZsq{}}\PY{l+s+s1}{delta\PYZus{}col}\PY{l+s+s1}{\PYZsq{}}\PY{p}{]} \PY{o}{=} \PY{l+s+s1}{\PYZsq{}}\PY{l+s+s1}{NaN}\PY{l+s+s1}{\PYZsq{}}
        \PY{n}{data}
\end{Verbatim}


    \section{How to delete a column}\label{how-to-delete-a-column}

    \begin{Verbatim}[commandchars=\\\{\}]
{\color{incolor}In [{\color{incolor} }]:} \PY{c+c1}{\PYZsh{} del data[\PYZsq{}name\PYZus{}of\PYZus{}the\PYZus{}col\PYZus{}to\PYZus{}delete\PYZsq{}]}
        
        \PY{k}{del} \PY{n}{data}\PY{p}{[}\PY{l+s+s1}{\PYZsq{}}\PY{l+s+s1}{delta\PYZus{}col}\PY{l+s+s1}{\PYZsq{}}\PY{p}{]}
        \PY{k}{print}\PY{p}{(}\PY{n}{data}\PY{p}{)}
\end{Verbatim}


    \section{How to drop a column?}\label{how-to-drop-a-column}

    \begin{Verbatim}[commandchars=\\\{\}]
{\color{incolor}In [{\color{incolor} }]:} \PY{n}{newdata} \PY{o}{=} \PY{n}{data}\PY{o}{.}\PY{n}{drop}\PY{p}{(}\PY{l+m+mi}{2}\PY{p}{)}
        \PY{k}{print}\PY{p}{(}\PY{n}{newdata}\PY{p}{)}
\end{Verbatim}


    \section{How to do a transpose of a
dataframe?}\label{how-to-do-a-transpose-of-a-dataframe}

    \begin{Verbatim}[commandchars=\\\{\}]
{\color{incolor}In [{\color{incolor} }]:} \PY{n}{dft} \PY{o}{=} \PY{n}{data}\PY{o}{.}\PY{n}{T} \PY{c+c1}{\PYZsh{}Transpose operation will interchange the rows and columns}
        \PY{n}{dft}
\end{Verbatim}

You can pass a number of data structures to DataFrame such as a ndarray, lists, dict, Series, and another DataFrame. You can also reindex to confirm to data to a new index. Reindexing is a powerful feature that allows you to access data in a number of different ways, and also to confirm data to some new time series or other index.
    \section{How to reindex the data?}\label{how-to-reindex-the-data}

    \begin{Verbatim}[commandchars=\\\{\}]
{\color{incolor}In [{\color{incolor} }]:} \PY{n}{new\PYZus{}data} \PY{o}{=} \PY{n}{data}\PY{o}{.}\PY{n}{reindex}\PY{p}{(}\PY{n}{index}\PY{o}{=}\PY{p}{[}\PY{l+m+mi}{0}\PY{p}{,}\PY{l+m+mi}{2}\PY{p}{]}\PY{p}{,} \PY{n}{columns}\PY{o}{=}\PY{p}{[}\PY{l+s+s1}{\PYZsq{}}\PY{l+s+s1}{company}\PY{l+s+s1}{\PYZsq{}}\PY{p}{,} \PY{l+s+s1}{\PYZsq{}}\PY{l+s+s1}{price}\PY{l+s+s1}{\PYZsq{}}\PY{p}{]}\PY{p}{)}
        \PY{k}{print}\PY{p}{(}\PY{n}{new\PYZus{}data}\PY{p}{)}
\end{Verbatim}


    \section{How to fill a missing value with some value in the data
frame?}\label{how-to-fill-a-missing-value-with-some-value-in-the-data-frame}

    \begin{Verbatim}[commandchars=\\\{\}]
{\color{incolor}In [{\color{incolor} }]:} \PY{n}{years1} \PY{o}{=} \PY{p}{[}\PY{l+m+mi}{90}\PY{p}{,} \PY{l+m+mi}{91}\PY{p}{,} \PY{l+m+mi}{92}\PY{p}{,} \PY{l+m+mi}{93}\PY{p}{,} \PY{l+m+mi}{94}\PY{p}{,} \PY{l+m+mi}{95}\PY{p}{]}
        \PY{n}{f4} \PY{o}{=} \PY{p}{\PYZob{}}\PY{l+m+mi}{90}\PY{p}{:}\PY{l+m+mi}{8}\PY{p}{,} \PY{l+m+mi}{91}\PY{p}{:}\PY{l+m+mi}{9}\PY{p}{,} \PY{l+m+mi}{92}\PY{p}{:}\PY{l+m+mi}{7}\PY{p}{,} \PY{l+m+mi}{93}\PY{p}{:}\PY{l+m+mi}{8}\PY{p}{,} \PY{l+m+mi}{94}\PY{p}{:}\PY{l+m+mi}{9}\PY{p}{,} \PY{l+m+mi}{95}\PY{p}{:}\PY{l+m+mi}{11}\PY{p}{\PYZcb{}}
        \PY{n}{firm4} \PY{o}{=} \PY{n}{pd}\PY{o}{.}\PY{n}{Series}\PY{p}{(}\PY{n}{f4}\PY{p}{,}\PY{n}{index}\PY{o}{=}\PY{n}{years}\PY{p}{)}
        \PY{n}{f5} \PY{o}{=} \PY{p}{\PYZob{}}\PY{l+m+mi}{90}\PY{p}{:}\PY{l+m+mi}{14}\PY{p}{,}\PY{l+m+mi}{91}\PY{p}{:}\PY{l+m+mi}{12}\PY{p}{,} \PY{l+m+mi}{92}\PY{p}{:}\PY{l+m+mi}{9}\PY{p}{,} \PY{l+m+mi}{93}\PY{p}{:}\PY{l+m+mi}{13}\PY{p}{,} \PY{l+m+mi}{94}\PY{p}{:}\PY{l+m+mi}{5}\PY{p}{,} \PY{l+m+mi}{95}\PY{p}{:}\PY{l+m+mi}{8}\PY{p}{\PYZcb{}}
        \PY{n}{firm5} \PY{o}{=} \PY{n}{pd}\PY{o}{.}\PY{n}{Series}\PY{p}{(}\PY{n}{f5}\PY{p}{,}\PY{n}{index}\PY{o}{=}\PY{n}{years}\PY{p}{)}
        \PY{n}{f6} \PY{o}{=} \PY{p}{\PYZob{}}\PY{l+m+mi}{90}\PY{p}{:}\PY{l+m+mi}{8}\PY{p}{,} \PY{l+m+mi}{91}\PY{p}{:} \PY{l+m+mi}{9}\PY{p}{,} \PY{l+m+mi}{92}\PY{p}{:}\PY{l+m+mi}{9}\PY{p}{,}\PY{l+m+mi}{93}\PY{p}{:}\PY{l+m+mi}{10}\PY{p}{,} \PY{l+m+mi}{94}\PY{p}{:}\PY{l+m+mi}{12}\PY{p}{,} \PY{l+m+mi}{95}\PY{p}{:} \PY{l+m+mi}{13}\PY{p}{\PYZcb{}}
        \PY{n}{firm6} \PY{o}{=} \PY{n}{pd}\PY{o}{.}\PY{n}{Series}\PY{p}{(}\PY{n}{f6}\PY{p}{,}\PY{n}{index}\PY{o}{=}\PY{n}{years}\PY{p}{)}
        \PY{n}{df2} \PY{o}{=} \PY{n}{pd}\PY{o}{.}\PY{n}{DataFrame}\PY{p}{(}\PY{n}{columns}\PY{o}{=}\PY{p}{[}\PY{l+s+s1}{\PYZsq{}}\PY{l+s+s1}{Firm1}\PY{l+s+s1}{\PYZsq{}}\PY{p}{,}\PY{l+s+s1}{\PYZsq{}}\PY{l+s+s1}{Firm2}\PY{l+s+s1}{\PYZsq{}}\PY{p}{,}\PY{l+s+s1}{\PYZsq{}}\PY{l+s+s1}{Firm3}\PY{l+s+s1}{\PYZsq{}}\PY{p}{]}\PY{p}{,}\PY{n}{index}\PY{o}{=}\PY{n}{years1}\PY{p}{)}
        \PY{n}{df2}\PY{o}{.}\PY{n}{Firm1} \PY{o}{=} \PY{n}{firm4}
        \PY{n}{df2}\PY{o}{.}\PY{n}{Firm2} \PY{o}{=} \PY{n}{firm5}
        \PY{n}{df2}\PY{o}{.}\PY{n}{Firm3} \PY{o}{=} \PY{n}{firm6}
        \PY{n}{df2}
\end{Verbatim}


    \begin{Verbatim}[commandchars=\\\{\}]
{\color{incolor}In [{\color{incolor} }]:} \PY{c+c1}{\PYZsh{}Note: reindex with only row arguments i.e we want row 88, 89 etc from above df2}
        \PY{n}{reindexdf2} \PY{o}{=} \PY{n}{df2}\PY{o}{.}\PY{n}{reindex}\PY{p}{(}\PY{p}{[}\PY{l+m+mi}{88}\PY{p}{,}\PY{l+m+mi}{89}\PY{p}{,}\PY{l+m+mi}{90}\PY{p}{,}\PY{l+m+mi}{91}\PY{p}{,}\PY{l+m+mi}{92}\PY{p}{,}\PY{l+m+mi}{93}\PY{p}{,}\PY{l+m+mi}{94}\PY{p}{,}\PY{l+m+mi}{95}\PY{p}{,}\PY{l+m+mi}{96}\PY{p}{,}\PY{l+m+mi}{97}\PY{p}{,}\PY{l+m+mi}{98}\PY{p}{]}\PY{p}{,} \PY{n}{fill\PYZus{}value}\PY{o}{=}\PY{l+m+mi}{0}\PY{p}{)}
        \PY{n}{reindexdf2}
\end{Verbatim}


    \section{The reason we have zeros is due to the fact that row 88, 89,
86,97 and 97
are}\label{the-reason-we-have-zeros-is-due-to-the-fact-that-row-88-89-8697-and-97-are}

\section{not present in our original data frame
df2}\label{not-present-in-our-original-data-frame-df2}

    \section{How to fill the missing value with the previous
value?}\label{how-to-fill-the-missing-value-with-the-previous-value}

    \begin{Verbatim}[commandchars=\\\{\}]
{\color{incolor}In [{\color{incolor} }]:} \PY{n}{reindexdf3} \PY{o}{=} \PY{n}{df2}\PY{o}{.}\PY{n}{reindex}\PY{p}{(}\PY{p}{[}\PY{l+m+mi}{88}\PY{p}{,}\PY{l+m+mi}{89}\PY{p}{,}\PY{l+m+mi}{90}\PY{p}{,}\PY{l+m+mi}{91}\PY{p}{,}\PY{l+m+mi}{92}\PY{p}{,}\PY{l+m+mi}{93}\PY{p}{,}\PY{l+m+mi}{94}\PY{p}{,}\PY{l+m+mi}{95}\PY{p}{,}\PY{l+m+mi}{96}\PY{p}{,}\PY{l+m+mi}{97}\PY{p}{,}\PY{l+m+mi}{98}\PY{p}{]}\PY{p}{,} \PY{n}{method}\PY{o}{=}\PY{l+s+s1}{\PYZsq{}}\PY{l+s+s1}{ffill}\PY{l+s+s1}{\PYZsq{}}\PY{p}{)}
        \PY{n}{reindexdf3}
\end{Verbatim}


    \section{Similarly, you have backfill (bfill) method to fill values
backwards.}\label{similarly-you-have-backfill-bfill-method-to-fill-values-backwards.}

    \section{WAP - In class exe :Create a data frame with three columns
named one, two and three and fill the values with random
numbers?}\label{wap---in-class-exe-create-a-data-frame-with-three-columns-named-one-two-and-three-and-fill-the-values-with-random-numbers}

\begin{verbatim}
Hint: Use numpy to create random numbers.
\end{verbatim}

    \section{Renaming the columns and rows of an existing data
frame}\label{renaming-the-columns-and-rows-of-an-existing-data-frame}

    \begin{Verbatim}[commandchars=\\\{\}]
{\color{incolor}In [{\color{incolor} }]:} \PY{k}{print}\PY{p}{(}\PY{l+s+s2}{\PYZdq{}}\PY{l+s+s2}{ }\PY{l+s+se}{\PYZbs{}n}\PY{l+s+s2}{ Before renaming the above dataframe }\PY{l+s+se}{\PYZbs{}n}\PY{l+s+s2}{\PYZdq{}}\PY{p}{)}
        
        
        \PY{k}{print}\PY{p}{(}\PY{n}{data}\PY{p}{)} \PY{c+c1}{\PYZsh{} Let\PYZsq{}s see the data frame which we have already created }
        \PY{n}{new\PYZus{}data} \PY{o}{=} \PY{n}{data}\PY{o}{.}\PY{n}{rename}\PY{p}{(}\PY{n}{columns}\PY{o}{=}\PY{p}{\PYZob{}}\PY{l+s+s1}{\PYZsq{}}\PY{l+s+s1}{ticker}\PY{l+s+s1}{\PYZsq{}} \PY{p}{:} \PY{l+s+s1}{\PYZsq{}}\PY{l+s+s1}{abbriviation}\PY{l+s+s1}{\PYZsq{}}\PY{p}{\PYZcb{}}\PY{p}{,}
        \PY{n}{index} \PY{o}{=} \PY{p}{\PYZob{}}\PY{l+m+mi}{0} \PY{p}{:} \PY{l+s+s1}{\PYZsq{}}\PY{l+s+s1}{company1}\PY{l+s+s1}{\PYZsq{}}\PY{p}{,} \PY{l+m+mi}{1} \PY{p}{:} \PY{l+s+s1}{\PYZsq{}}\PY{l+s+s1}{company2}\PY{l+s+s1}{\PYZsq{}}\PY{p}{,} \PY{l+m+mi}{2} \PY{p}{:} \PY{l+s+s1}{\PYZsq{}}\PY{l+s+s1}{company3}\PY{l+s+s1}{\PYZsq{}}\PY{p}{,}\PY{l+m+mi}{3}\PY{p}{:}\PY{l+s+s1}{\PYZsq{}}\PY{l+s+s1}{company4}\PY{l+s+s1}{\PYZsq{}}\PY{p}{,}\PY{l+m+mi}{4}\PY{p}{:}\PY{l+s+s1}{\PYZsq{}}\PY{l+s+s1}{company5}\PY{l+s+s1}{\PYZsq{}}\PY{p}{\PYZcb{}}\PY{p}{)}
        
        \PY{k}{print}\PY{p}{(}\PY{l+s+s2}{\PYZdq{}}\PY{l+s+s2}{ }\PY{l+s+se}{\PYZbs{}n}\PY{l+s+s2}{ After renaming the above dataframe }\PY{l+s+se}{\PYZbs{}n}\PY{l+s+s2}{\PYZdq{}}\PY{p}{)}
        
        \PY{k}{print}\PY{p}{(}\PY{n}{new\PYZus{}data}\PY{p}{)}
\end{Verbatim}


    \section{Reading from a CSV file and creating a data
frame}\label{reading-from-a-csv-file-and-creating-a-data-frame}

    \begin{Verbatim}[commandchars=\\\{\}]
{\color{incolor}In [{\color{incolor} }]:} \PY{n}{mycars} \PY{o}{=} \PY{n}{pd}\PY{o}{.}\PY{n}{read\PYZus{}csv}\PY{p}{(}\PY{l+s+s1}{\PYZsq{}}\PY{l+s+s1}{motor\PYZus{}cars.csv}\PY{l+s+s1}{\PYZsq{}}\PY{p}{)}
        \PY{n}{mycars}\PY{o}{.}\PY{n}{head}\PY{p}{(}\PY{p}{)}
        
        \PY{c+c1}{\PYZsh{} https://stat.ethz.ch/R\PYZhy{}manual/R\PYZhy{}devel/library/datasets/html/mtcars.html}
        \PY{c+c1}{\PYZsh{} The above link would help you to understand the data set }
\end{Verbatim}


    \section{Iterating the data frame}\label{iterating-the-data-frame}

    \begin{Verbatim}[commandchars=\\\{\}]
{\color{incolor}In [{\color{incolor} }]:} \PY{c+c1}{\PYZsh{} Lets rename the first column (scroll up for the see the default name)}
        \PY{n}{mycars}\PY{o}{.}\PY{n}{rename}\PY{p}{(}\PY{n}{columns}\PY{o}{=}\PY{p}{\PYZob{}}\PY{l+s+s1}{\PYZsq{}}\PY{l+s+s1}{Unnamed: 0}\PY{l+s+s1}{\PYZsq{}} \PY{p}{:} \PY{l+s+s1}{\PYZsq{}}\PY{l+s+s1}{car\PYZus{}model}\PY{l+s+s1}{\PYZsq{}}\PY{p}{\PYZcb{}}\PY{p}{,} \PY{n}{inplace}\PY{o}{=}\PY{n+nb+bp}{True}\PY{p}{)}
        
        \PY{c+c1}{\PYZsh{} The below code will print all the column names in the data frame }
        \PY{k}{for} \PY{n}{col} \PY{o+ow}{in} \PY{n}{mycars}\PY{p}{:}
           \PY{k}{print}\PY{p}{(}\PY{n}{col}\PY{p}{)}
\end{Verbatim}


    \section{understanding iteritems()}\label{understanding-iteritems}

    \begin{Verbatim}[commandchars=\\\{\}]
{\color{incolor}In [{\color{incolor} }]:} \PY{n}{headcars} \PY{o}{=} \PY{n}{mycars}\PY{o}{.}\PY{n}{head}\PY{p}{(}\PY{p}{)}
        
        
        \PY{k}{for} \PY{n}{key}\PY{p}{,}\PY{n}{value} \PY{o+ow}{in} \PY{n}{headcars}\PY{o}{.}\PY{n}{iteritems}\PY{p}{(}\PY{p}{)}\PY{p}{:}
           \PY{k}{print} \PY{p}{(}\PY{n}{key}\PY{p}{,}\PY{n}{value}\PY{p}{)}
        
        \PY{c+c1}{\PYZsh{}The iteritems is used to print the column items as key value pairs }
        \PY{c+c1}{\PYZsh{} i.e col name is the key and the data items are the values }
\end{Verbatim}


    \section{Understanding iterrows()}\label{understanding-iterrows}

    \begin{Verbatim}[commandchars=\\\{\}]
{\color{incolor}In [{\color{incolor} }]:} \PY{c+c1}{\PYZsh{} All the column elements will be displayed for every row }
        \PY{k}{for} \PY{n}{row\PYZus{}index}\PY{p}{,}\PY{n}{row} \PY{o+ow}{in} \PY{n}{headcars}\PY{o}{.}\PY{n}{iterrows}\PY{p}{(}\PY{p}{)}\PY{p}{:}
             \PY{k}{print}\PY{p}{(}\PY{n}{row\PYZus{}index}\PY{p}{,}\PY{n}{row}\PY{p}{)}
\end{Verbatim}


    \section{WAP : In class exe: Modify the above program to print the
output only
for}\label{wap-in-class-exe-modify-the-above-program-to-print-the-output-only-for}

row\_index 3 and 4

    \section{Sorting}\label{sorting}

    \begin{Verbatim}[commandchars=\\\{\}]
{\color{incolor}In [{\color{incolor} }]:} \PY{n}{sorted\PYZus{}cars}\PY{o}{=}\PY{n}{headcars}\PY{o}{.}\PY{n}{sort\PYZus{}index}\PY{p}{(}\PY{n}{ascending}\PY{o}{=}\PY{n+nb+bp}{False}\PY{p}{)}
        \PY{k}{print}\PY{p}{(}\PY{n}{sorted\PYZus{}cars}\PY{p}{)}
        
        \PY{c+c1}{\PYZsh{}Note:Sort index will sort the data frame based on the row index. Remove the }
        \PY{c+c1}{\PYZsh{} ascending = False option and observe the output }
\end{Verbatim}


    \section{Sorting based on column
values}\label{sorting-based-on-column-values}

    \begin{Verbatim}[commandchars=\\\{\}]
{\color{incolor}In [{\color{incolor} }]:} \PY{c+c1}{\PYZsh{} We are going to sort the data frame by the milage (mpg) column }
        \PY{n}{topmpg} \PY{o}{=} \PY{n}{mycars}\PY{o}{.}\PY{n}{sort\PYZus{}values}\PY{p}{(}\PY{n}{by}\PY{o}{=}\PY{l+s+s1}{\PYZsq{}}\PY{l+s+s1}{mpg}\PY{l+s+s1}{\PYZsq{}}\PY{p}{)} 
        \PY{n}{topmpg}
        
        \PY{c+c1}{\PYZsh{} Note that the row index is not in sorted order when we sort based on a column value }
        \PY{c+c1}{\PYZsh{} We can now use sort\PYZus{}index() to sort the data frame in the ascending order of row index }
\end{Verbatim}


    \section{WAP : (In class exe) To pick only the top 10 cars with maximum
milage}\label{wap-in-class-exe-to-pick-only-the-top-10-cars-with-maximum-milage}

    \section{Indexing and selecting data}\label{indexing-and-selecting-data}

    \begin{Verbatim}[commandchars=\\\{\}]
{\color{incolor}In [{\color{incolor} }]:} \PY{n}{modelnames} \PY{o}{=} \PY{n}{mycars}\PY{o}{.}\PY{n}{loc}\PY{p}{[}\PY{p}{:}\PY{p}{,}\PY{l+s+s1}{\PYZsq{}}\PY{l+s+s1}{car\PYZus{}model}\PY{l+s+s1}{\PYZsq{}}\PY{p}{]} \PY{c+c1}{\PYZsh{} pick all the rows of the column \PYZdq{}car\PYZus{}model\PYZdq{}}
        \PY{k}{print}\PY{p}{(}\PY{n}{modelnames}\PY{o}{.}\PY{n}{head}\PY{p}{(}\PY{p}{)}\PY{p}{)}
        
        \PY{k}{print}\PY{p}{(}\PY{l+s+s2}{\PYZdq{}}\PY{l+s+se}{\PYZbs{}n}\PY{l+s+s2}{ }\PY{l+s+se}{\PYZbs{}n}\PY{l+s+s2}{\PYZdq{}}\PY{p}{)}
        
        \PY{n}{name\PYZus{}mpg} \PY{o}{=} \PY{n}{mycars}\PY{o}{.}\PY{n}{loc}\PY{p}{[}\PY{p}{:}\PY{p}{,}\PY{p}{[}\PY{l+s+s1}{\PYZsq{}}\PY{l+s+s1}{car\PYZus{}model}\PY{l+s+s1}{\PYZsq{}}\PY{p}{,}\PY{l+s+s1}{\PYZsq{}}\PY{l+s+s1}{mpg}\PY{l+s+s1}{\PYZsq{}}\PY{p}{]}\PY{p}{]} \PY{c+c1}{\PYZsh{} We can specify more than one column }
        \PY{k}{print}\PY{p}{(}\PY{n}{name\PYZus{}mpg}\PY{o}{.}\PY{n}{head}\PY{p}{(}\PY{p}{)}\PY{p}{)}
        
        \PY{k}{print}\PY{p}{(}\PY{l+s+s2}{\PYZdq{}}\PY{l+s+se}{\PYZbs{}n}\PY{l+s+s2}{ }\PY{l+s+se}{\PYZbs{}n}\PY{l+s+s2}{\PYZdq{}}\PY{p}{)}
        
        
        \PY{n}{rows\PYZus{}name\PYZus{}mpg} \PY{o}{=} \PY{n}{mycars}\PY{o}{.}\PY{n}{loc}\PY{p}{[}\PY{p}{[}\PY{l+m+mi}{0}\PY{p}{,}\PY{l+m+mi}{1}\PY{p}{,}\PY{l+m+mi}{2}\PY{p}{]}\PY{p}{,}\PY{p}{[}\PY{l+s+s1}{\PYZsq{}}\PY{l+s+s1}{car\PYZus{}model}\PY{l+s+s1}{\PYZsq{}}\PY{p}{,}\PY{l+s+s1}{\PYZsq{}}\PY{l+s+s1}{mpg}\PY{l+s+s1}{\PYZsq{}}\PY{p}{]}\PY{p}{]}
        \PY{k}{print}\PY{p}{(}\PY{n}{rows\PYZus{}name\PYZus{}mpg}\PY{p}{)}
        
        \PY{c+c1}{\PYZsh{} The above command mycars.loc[[0,1,2],[\PYZsq{}car\PYZus{}model\PYZsq{},\PYZsq{}mpg\PYZsq{}]] can also be executed like below eg }
        \PY{c+c1}{\PYZsh{} mycars.loc[[0:2],[\PYZsq{}car\PYZus{}model\PYZsq{},\PYZsq{}mpg\PYZsq{}]]}
\end{Verbatim}


    \section{Indexing the data frames using
iloc()}\label{indexing-the-data-frames-using-iloc}

    \begin{Verbatim}[commandchars=\\\{\}]
{\color{incolor}In [{\color{incolor} }]:} \PY{k}{print}\PY{p}{(}\PY{n}{mycars}\PY{o}{.}\PY{n}{iloc}\PY{p}{[}\PY{p}{[}\PY{l+m+mi}{1}\PY{p}{,}\PY{l+m+mi}{2}\PY{p}{]}\PY{p}{,} \PY{p}{[}\PY{l+m+mi}{3}\PY{p}{,}\PY{l+m+mi}{4}\PY{p}{]}\PY{p}{]}\PY{p}{)}
              \PY{c+c1}{\PYZsh{} Pick row 1\PYZam{}2 elements for column 3 and 4 }
\end{Verbatim}


    \section{Try these options in iloc()}\label{try-these-options-in-iloc}

    \begin{Verbatim}[commandchars=\\\{\}]
{\color{incolor}In [{\color{incolor} }]:} \PY{k}{print}\PY{p}{(}\PY{n}{mycars}\PY{o}{.}\PY{n}{iloc}\PY{p}{[}\PY{l+m+mi}{1}\PY{p}{:}\PY{l+m+mi}{3}\PY{p}{,} \PY{p}{:}\PY{p}{]}\PY{p}{)} 
        \PY{c+c1}{\PYZsh{} Pick all cols for row 1 to 3 }
        
        \PY{k}{print}\PY{p}{(}\PY{n}{mycars}\PY{o}{.}\PY{n}{iloc}\PY{p}{[}\PY{p}{:}\PY{p}{,} \PY{l+m+mi}{1}\PY{p}{:}\PY{l+m+mi}{3}\PY{p}{]}\PY{p}{)} 
        \PY{c+c1}{\PYZsh{} pick all rows of col 1 to 3 }
\end{Verbatim}


    \section{Understanding .ix() function for slicing a data
frame}\label{understanding-.ix-function-for-slicing-a-data-frame}

    \begin{Verbatim}[commandchars=\\\{\}]
{\color{incolor}In [{\color{incolor} }]:} \PY{k}{print}\PY{p}{(}\PY{n}{mycars}\PY{o}{.}\PY{n}{ix}\PY{p}{[}\PY{p}{:}\PY{l+m+mi}{3}\PY{p}{]}\PY{p}{)}
        \PY{c+c1}{\PYZsh{} The above is used to print the }
\end{Verbatim}


    \section{Try running the below piece of code to understand different
ways of using .ix()
function}\label{try-running-the-below-piece-of-code-to-understand-different-ways-of-using-.ix-function}

    \begin{Verbatim}[commandchars=\\\{\}]
{\color{incolor}In [{\color{incolor} }]:} \PY{k}{print}\PY{p}{(}\PY{n}{mycars}\PY{o}{.}\PY{n}{ix}\PY{p}{[}\PY{p}{:}\PY{p}{,}\PY{l+s+s1}{\PYZsq{}}\PY{l+s+s1}{mpg}\PY{l+s+s1}{\PYZsq{}}\PY{p}{]}\PY{p}{)}
\end{Verbatim}


    \section{Take home exercise}\label{take-home-exercise}

\section{Description about the "mtcars" data set can be found in the
below
link}\label{description-about-the-mtcars-data-set-can-be-found-in-the-below-link}

https://stat.ethz.ch/R-manual/R-devel/library/datasets/html/mtcars.html

Create a new data frame from the "mtcars" provided to you as csv file.
The new data frame must have the following colunms

Col 1 : Cubic capacity in cubic centemeters and this must be a whole
number ( with floor and ceiling corrected based on \textgreater{} or
\textless{} 0.5 respectively Hint : Use the round() function) the
existing data frame contains this data in cubic inches (engine
diplacement column)

1 cubic inch = 16.387 cubic centimeteres

Col 2 : Power is to Weight Ratio, weight of the car is provided in units
per 1000 LBS, you would need to convert it to LB's first and then
calculate the power/weight ratio. In case if this is in too low a
decimal number, then you will have to represent it appropriately by
convering it to a whole number which is readable.

Col 3 : Milage (Note: The places where NaN is marked have to be ignored
for calculation and should be present as NaN in the final output)

The final output in the data frame is must be sorted based on the cc of
the engine. The final output must contain the rows of only those car
models where is engine capacity is greater than 2500 (where cc is
converted from cubic inches to cubic centimeters)

The code should be well organized into user defined functions wherever
applicable. The car names columns must be retained as the original data
frame.


    % Add a bibliography block to the postdoc
    
    
    
    \end{document}
